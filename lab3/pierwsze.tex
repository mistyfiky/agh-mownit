\paragraph{Treść}~\\
Proszę zastosować aproksymację Czebyszewa dla funkcji:\\
1. $ y = exp(x**2) $ w przedziale od -1 do 1\\
2. $ y = abs(x+x**3) $ w przedziale od -1 do 1\\
3. $ y = sign(x) $ w przedziale od -1 do 1\\
Do aproksymacji należy użyć biblioteki GSL.\\
Dla każdej funkcji proszę:\\
- narysować wykres funkcji aproksymowanej i aproksymującej (gnuplot)\\
- sprawdzić, jak wynik zależy od stopnia wielomianu aproksymującego - przedstawić odpowiednie wykresy\\

\lstinputlisting[caption=Kod programu do zadania pierwszego,label=lst:PierwszeC]{pierwsze.c}

\lstinputlisting[caption=Komendy dla programu gnuplot do zadania pierwszego podpunkt 1,label=lst:Pierwsze1Txt]{pierwsze1.txt}

\lstinputlisting[caption=Komendy dla programu gnuplot do zadania pierwszego podpunkt 2,label=lst:Pierwsze2Txt]{pierwsze2.txt}

\lstinputlisting[caption=Komendy dla programu gnuplot do zadania pierwszego podpunkt 3,label=lst:Pierwsze3Txt]{pierwsze3.txt}

\begin{figure}[p]
  \caption{Wynik programu gnuplot na podstawie danych z programu do zadania pierwszego podpunkt 1}
  \label{fig:Pierwsze1Jpg}
  \centering
  \includegraphics[width=\textwidth]{pierwsze1.jpg}
\end{figure}

\begin{figure}[p]
  \caption{Wynik programu gnuplot na podstawie danych z programu do zadania pierwszego podpunkt 2}
  \label{fig:Pierwsze2Jpg}
  \centering
  \includegraphics[width=\textwidth]{pierwsze2.jpg}
\end{figure}

\begin{figure}[p]
  \caption{Wynik programu gnuplot na podstawie danych z programu do zadania pierwszego podpunkt 3}
  \label{fig:Pierwsze3Jpg}
  \centering
  \includegraphics[width=\textwidth]{pierwsze3.jpg}
\end{figure}

\paragraph{Podsumowanie}~\\
Kod programu do zadania pierwszego widoczny jest na listowaniu~\ref{lst:PierwszeC} został przygotowany na podstawie przykłądu z dokumentacji biblioteki gsl~\cite{ChebyshevApproximationsGSL25Documentation}.
Generuje on pliki z wartościami wielomianu Czybyszewa dziesiątego i czterdziestego stopnia dla funkcji z treści zadania dla podanego przedziału, które następnie przekazywane są do progamu gnuplot, poprzez załadowanie do niego komend z listowania~\ref{lst:Pierwsze1Txt}, ~\ref{lst:Pierwsze2Txt} oraz ~\ref{lst:Pierwsze3Txt}.
\\
Wynik podpunktu 1 zaprezentowany jest na rysunku~\ref{fig:Pierwsze1Jpg}.
Kształty wykresów pokrywają się idealnie.
Funkcja ta daje się świetnie aproksymować dla obu stopni wielomianu Czybyszewa.
\\
Wynik podpunktu 2 zaprezentowany jest na rysunku~\ref{fig:Pierwsze2Jpg}.
W przypadku tej funkcji problemem jest charakterystyczne wcięcie aproksymowanej funkcji.
Nie radzi sobie z nim wielomian niższego stopnia, jednak ten wyższego, dobrze odwzorowuje przebieg funkcji.
\\
Wynik podpunktu 3 zaprezentowany jest na rysunku~\ref{fig:Pierwsze3Jpg}.
W tym przypadku nieciągłość funkcji powoduje duże nieścisłości wielomianu Czybyszewa.
Szczególnie widać w okolicach zera.
Wielomian niższego stopnia "wolno" dokonuje zmiany znaku.
Trochę lepiej radzi sobie z tym wielomian wyższego stopnia - zmiana następuje "szybciej" i mniejsze są oscylacje wartości pod i nad osią x.
\\
Jak widać na przykładach, zwiększając stopień wielomianu uzyskujemy większą dokładność przybliżeń funkcji.

\begin{thebibliography}{9}
  \bibitem{ChebyshevApproximationsGSL25Documentation} Chebyshev Approximations — GSL 2.5 documentation https://www.gnu.org/software/gsl/doc/html/cheb.html\#examples
\end{thebibliography}
