\paragraph{Treść}~\\
\import{./}{zad}\\
Obliczyć wartość całki korzystając z funkcji gsl\_integration\_qag metodą GSL\_INTEG\_GAUSS15 dla zadanych dokładności takich jak w p.3.
Sprawdzić, ile przedziałów (intervals) potrzebuje ta procedura aby osiągnąc zadaną dokładność (1e-3, 1e-4, 1e-5 i 1e-6).
Porownać, ile przedziałów potrzebuje metoda prostokątów do osiągnięcia podobnej dokładności.
Patrz przykład w dokumentacji GSL~\cite{NumericalIntegrationGSL25Documentation}.

\lstinputlisting[caption=Kod programu do zadania 4,label=lst:Zad4C]{zad4.c}

\lstinputlisting[caption=Rezultat programu do zadania 4,label=lst:Zad4Out]{zad4.out}

\paragraph{Podsumowanie}~\\
Program~\ref{lst:Zad4C} używa biblioteki gsl do obliczenia całek z funkcji podanych w treści zadania.
Działa on znacznie wydajniej niż metoda prostokątów i już przy 1 przedziale w przypadku pierwszej funkcji oraz 6 w przypadku drugiej znajduje rozwiązania z błędem poniżej 1e-7, co można wyczytać z rezultatu jego działania~\ref{lst:Zad4Out}.

\begin{thebibliography}{9}
  \bibitem{NumericalIntegrationGSL25Documentation} Numerical Integration — GSL 2.5 documentation https://www.gnu.org/software/gsl/doc/html/integration.html\#adaptive-integration-example
\end{thebibliography}
