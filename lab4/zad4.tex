\paragraph{Treść}~\\
Obliczyć wartość całki korzystając z funkcji gsl\_integration\_qag metodą GSL\_INTEG\_GAUSS15 dla zadanych dokładności takich jak w p.3.
Sprawdzić, ile przedziałów (intervals) potrzebuje ta procedura aby osiągnąc zadaną dokładność (1e-3, 1e-4, 1e-5 i 1e-6).
Porownać, ile przedziałów potrzebuje metoda prostokątów do osiągnięcia podobnej dokładności.
Patrz przykład w dokumentacji GSL~\cite{NumericalIntegrationGSL25Documentation}.

\lstinputlisting[caption=Kod programu do zadania 4,label=lst:Zad4C]{zad4.c}

\paragraph{Podsumowanie}~\\
TODO

\begin{thebibliography}{9}
  \bibitem{NumericalIntegrationGSL25Documentation} Numerical Integration — GSL 2.5 documentation https://www.gnu.org/software/gsl/doc/html/integration.html\#adaptive-integration-example
\end{thebibliography}
