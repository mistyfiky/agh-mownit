\paragraph{Treść}~\\
\import{./}{zad}\\
Zbadać, przy użyciu programu z poprzedniego punktu, jak zmienia się błąd całkowania wraz ze wzrostem liczby podprzedziałów.
Kiedy błąd jest mniejszy niż 1e-3, 1e-4, 1e-5 i 1e-6?

\lstinputlisting[caption=Skrypt do generowania danych do zadania 3,label=lst:Zad3Sh]{zad3.sh}

\lstinputlisting[caption=Komendy dla programu gnuplot do zadania 3,label=lst:Zad3Txt]{zad3.txt}

\begin{figure}[p]
  \caption{Wynik programu gnuplot na podstawie danych z programu do zadania 2}
  \label{fig:Zad3Jpg}
  \centering
  \includegraphics[width=\textwidth]{zad3.jpg}
\end{figure}

\paragraph{Podsumowanie}~\\
Przy pomocy programu z zadania 2~\ref{lst:Zad2C}, skryptu~\ref{lst:Zad3Sh} oraz komend dla programu gnuplot~\ref{lst:Zad3Txt} został wygenerowany wykres~\ref{fig:Zad3Jpg}.
Można z niego odczytać, że aby błąd całkowania za pomocą metody prostokątów był mniejszy niż 1e-3, 1e-4, 1e-5 i 1e-6 trzeba odpowienio dla funkcji $ x^2 $ zwiększyć liczbę przedziałów do pomiędzy 10 a 100, 100, trochę powyżej 100 i 1000.
Dla funkcji $ \frac{1}{\sqrt{x}} $ nie udało się zejść poniżej 1e-3 ze względu na zakres typu doule, jednak przy lepszym typie byłoby to ok 1e6.
