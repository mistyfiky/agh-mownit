\paragraph{Treść}~\\
\import{./}{zad}\\
Napisać program obliczającą całkę metodą prostokątów.
Program powinien mieć następujące parametry:\\
- float a - początek przedziału\\
- float b - koniec przedziału\\
- int n - ilość podprzedziałów, na które dzielimy przedział (a,b)

\lstinputlisting[caption=Kod programu do zadania 2,label=lst:Zad2C]{zad2.c}

\paragraph{Podsumowanie}~\\
Kod programu do zadania 2 widoczny jest na listowaniu~\ref{lst:Zad2C}.
Lekko polepsza on wyniki, ze względu na obliczanie pola trapezu zamiast prostokątu.
