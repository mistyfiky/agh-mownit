\paragraph{Treść}~\\
\import{./}{zad}\\
Napisać program obliczającą całkę metodą prostokątów.
Program powinien mieć następujące parametry:\\
- float a - początek przedziału\\
- float b - koniec przedziału\\
- int n - ilość podprzedziałów, na które dzielimy przedział (a,b)

\lstinputlisting[caption=Kod programu do zadania 2,label=lst:Zad2C]{zad2.c}

\paragraph{Podsumowanie}~\\
Kod programu do zadania 2 widoczny jest na listowaniu~\ref{lst:Zad2C}.
Oblicza on całkę oznaczoną danych 2 funkcji dwoma sposobami.
Pierwszy sposób lekko polepsza wyniki, ze względu na obliczanie pola trapezu zamiast prostokątu.
Drugi sposób daje dokładny wynik, ze względu na użycie ręcznego wyniku całki.
Program dodatkowo wyświetla różnicę w wynikach.
