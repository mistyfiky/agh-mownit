\paragraph{Treść}~\\
Korzystając z przykładu napisz program, który:\\
1. Jako parametr pobiera rozmiar układu równań n\\
2. Generuje macierz układu A(nxn) i wektor wyrazów wolnych b(n)\\
3. Rozwiązuje układ równań\\
4. Sprawdza poprawność rozwiązania (tj., czy Ax=b)\\
5. Mierzy czas dekompozycji macierzy - do mierzenia czasu można skorzystać z przykładowego programu dokonującego pomiaru czasu procesora spędzonego w danym fragmencie programu.\\
6. Mierzy czas rozwiązywania układu równań\\
Zadanie domowe: Narysuj wykres zależności czasu dekompozycji i czasu rozwiązywania układu od rozmiaru układu równań. Wykonaj pomiary dla 10 wartości z przedziału od 10 do 1000.

\lstinputlisting[caption=Program do zadania,label=lst:ZadC]{zad.c}

\lstinputlisting[caption=Skrypt do generowania danych do zadania,label=lst:ZadSh]{zad.sh}

\lstinputlisting[caption=Dane do zadania,label=lst:ZadDat]{zad.dat}

\lstinputlisting[caption=Komendy dla programu gnuplot do zadania,label=lst:ZadTxt]{zad.txt}

\begin{figure}[p]
  \caption{Wynik programu gnuplot na podstawie komend do zadania}
  \label{fig:ZadJpg}
  \centering
  \includegraphics[width=\textwidth]{zad.jpg}
\end{figure}
