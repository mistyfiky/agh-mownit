\documentclass{article}

\usepackage[polish]{babel}
\usepackage[utf8]{inputenc}
\usepackage{polski}
\usepackage[T1]{fontenc}
\frenchspacing
\usepackage{indentfirst}

\usepackage{listings}

\title{MOwNiT - sprawozdanie 1}
\input{../author.tex}
\input{date.tex}

\begin{document}

  \lstset{language=C}

  \maketitle
  \thispagestyle{empty}
  \newpage

  \paragraph{Zadanie 1.} ~\\
  Napisać program liczący kolejne wyrazy ciągu: \\
  $ x_{n+1} = x_{n} + 3.0 * x_{n} * (1 - x_{n}) $ \\
  startując z punktu $ x_{0} = 0.01 $.
  Wykonać to zadanie dla różnych reprezentacji liczb (\textit{float}, \textit{double}).
  Dlaczego wyniki się rozbiegają? \\
  \textit{Uwaga: Należy wprowadzić zmienne pomocnicze, aby uniknąć obliczeń w rejestrach procesora.} \\
  \lstinputlisting[caption={Kod programu do zadania 1}]{zad1.c}
  \newpage

  \paragraph{Zadanie 2.} ~\\
  Napisać program liczący ciąg z wcześniejszego zadania, ale wg. wzoru \\
  $ x_{n+1} = 4.0 * x_{n} - 3.0 * x_{n} * x_{n} $ \\
  - porównać z wynikami z wcześniejszego zadania. \\
  \lstinputlisting[caption={Kod programu do zadania 2}]{zad2.c}
  \newpage

  \paragraph{Zadanie 3.} ~\\
  Znaleźć ,,maszynowe epsilon'', czyli najmniejszą liczbę a, taką że $ a + 1 > 1 $ \\
  \lstinputlisting[caption={Kod programu do zadania 3}]{zad3.c}
  \newpage

\end{document}
