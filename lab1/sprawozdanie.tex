\documentclass{article}

\usepackage[polish]{babel}
\usepackage[utf8]{inputenc}
\usepackage{polski}
\usepackage[T1]{fontenc}
\frenchspacing
\usepackage{indentfirst}

\usepackage{listings}
\lstset{language=C}

\title{MOwNiT - sprawozdanie 1}
\input{../author.tex}
\input{date.tex}

\begin{document}

  \maketitle
  \thispagestyle{empty}


  \newpage
  \section*{Zadanie 1.} \label{sec:zadanie_1}

  \paragraph{Treść} ~\\
  Napisać program liczący kolejne wyrazy ciągu: \\
  $ x_{n+1} = x_{n} + 3.0 * x_{n} * (1 - x_{n}) $ \\
  startując z punktu $ x_{0} = 0.01 $.
  Wykonać to zadanie dla różnych reprezentacji liczb (\textit{float}, \textit{double}).
  Dlaczego wyniki się rozbiegają? \\
  \textit{Uwaga: Należy wprowadzić zmienne pomocnicze, aby uniknąć obliczeń w rejestrach procesora.} \\

  \paragraph{Rozwiązanie} ~\\
  \lstinputlisting[caption={Kod programu do zadania 1},captionpos=b]{zad1.c}

  \paragraph{Wnioski} ~\\


  \newpage
  \section*{Zadanie 2.} \label{sec:zadanie_2}

  \paragraph{Treść} ~\\
  Napisać program liczący ciąg z wcześniejszego zadania, ale wg. wzoru \\
  $ x_{n+1} = 4.0 * x_{n} - 3.0 * x_{n} * x_{n} $ \\
  - porównać z wynikami z wcześniejszego zadania. \\

  \paragraph{Rozwiązanie} ~\\
  \lstinputlisting[caption={Kod programu do zadania 2},captionpos=b]{zad2.c}

  \paragraph{Wnioski} ~\\


  \newpage
  \section*{Zadanie 3} \label{sec:zadanie_3}

  \paragraph{Treść} ~\\
  Znaleźć ,,maszynowe epsilon'', czyli najmniejszą liczbę a, taką że $ a + 1 > 1 $ \\

  \paragraph{Rozwiązanie} ~\\
  \lstinputlisting[caption={Kod programu do zadania 3},captionpos=b]{zad3.c}

  \paragraph{Wnioski} ~\\

\end{document}
