\paragraph{Treść}~\\
Znaleźć ,,maszynowe epsilon'', czyli najmniejszą liczbę a, taką że $ a + 1 > 1 $

\lstinputlisting[caption={Kod programu do zadania 3},label=lst:KodProgramuDoZadania3]{zad3.c}

\begin{table}[h]
  \caption{Wynik programu do zadania 3}
  \label{tab:WynikProgramuDoZadania3}
  \centering
  \begin{tabular}{c|c}
    float & double\\
    \hline 0.000000119209289550781250000000 & 0.000000000000000222044604925031\\
  \end{tabular}
\end{table}

\paragraph{Podsumowanie}~\\
Kod programu do zadania 3 zaprezentowany jest na listingu~\ref{lst:KodProgramuDoZadania3}.
Dzieli on liczbę na 2 tak długo, aż dojdzie do momentu, w którym potraktuje ją jako 0 (dodanie jej do 1 zwróci 1).
Fragmenty z wyników jego działania widać w tabeli~\ref{tab:WynikProgramuDoZadania3}.
Dla mojego komputera znalazł maszynowe epsilon pojedynczej (binary32) oraz podwójnej (binary64) precyzji zgodne ze standardem IEEE 754 - 2008~\cite{WikiMachineEpsilon}.

\begin{thebibliography}{9}
  \bibitem{WikiMachineEpsilon} Machine epsilon - Wikipedia https://en.wikipedia.org/wiki/Machine\_epsilon
\end{thebibliography}
