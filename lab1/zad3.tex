\paragraph{Treść} ~\\
Znaleźć ,,maszynowe epsilon'', czyli najmniejszą liczbę a, taką że $ a + 1 > 1 $ \\

\paragraph{Rozwiązanie} ~\\
\lstinputlisting[caption={Kod programu do zadania 3},label=lst:KodProgramuDoZadania3]{zad3.c} ~\\

\paragraph{Wynik} ~\\
\begin{table}[h]
  \centering
  \begin{tabular}{c|c}
    float & double \\
    \hline 0.000000119209289550781250000000 & 0.000000000000000222044604925031 \\
  \end{tabular}
  \caption{Wynik programu do zadania 3}
  \label{tab:WynikProgramuDoZadania3}
\end{table}

\paragraph{Podsumowanie} ~\\
Program dzieli liczbę na 2 tak długo, aż dojdzie do momentu, w którym potraktuje ją jako 0 (dodanie jej do 1 zwróci 1).
Dla mojego komputera znalazł maszynowe epsilon pojedynczej (binary32) oraz podwójnej (binary64) precyzji zgodne ze standardem IEEE 754 - 2008\nocite{WikiMachineEpsilon}.

\begin{thebibliography}{9}
  \bibitem{WikiMachineEpsilon} Machine epsilon - Wikipedia https://en.wikipedia.org/wiki/Machine\_epsilon
\end{thebibliography}
