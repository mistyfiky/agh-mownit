\paragraph{Treść} ~\\
Napisać program liczący ciąg z wcześniejszego zadania, ale wg. wzoru \\
$ x_{n+1} = 4.0 * x_{n} - 3.0 * x_{n} * x_{n} $ \\
- porównać z wynikami z wcześniejszego zadania. \\

\paragraph{Rozwiązanie} ~\\
\lstinputlisting[caption={Kod programu do zadania 2},label=lst:KodProgramuDoZadania2]{zad2.c} ~\\

\paragraph{Wynik} ~\\
\begin{table}[h]
  \centering
  \begin{tabular}{r|c|c}
    n & float & double \\
    \hline 0 & 0.010000 & 0.010000 \\
    \hline 1 & 0.039700 & 0.039700 \\
    \hline 2 & 0.154072 & 0.154072 \\
    \hline 3 & 0.545073 & 0.545073 \\
    \hline 4 & 1.288978 & 1.288978 \\
    \hline 5 & 0.171519 & 0.171519 \\
    \hline \textbf{6} & \textbf{0.597821} & \textbf{0.597820} \\
    \hline 7 & 1.319114 & 1.319114 \\
    \hline ... & ... & ... \\
    \hline 28 & 0.003118 & 0.850352 \\
    \hline \textbf{29} & \textbf{0.012443} & \textbf{1.232112} \\
    \hline 30 & 0.049309 & 0.374147 \\
    \hline ... & ... & ... \\
    \hline
  \end{tabular}
  \caption{Wynik programu do zadania 2}
  \label{tab:WynikProgramuDoZadania2}
\end{table}

\paragraph{Podsumowanie} ~\\
Podobnie jak w poprzednim programie, skompilowany kod liczy 99 kolejnych liczb ze wzoru podanego w treści zadania.
Zachodzi tutaj podobna sytuacja jak w poprzednim zadaniu.
Pierwszą różnicę widać już przy 6 iteracji pętli.
W tym przypadku, znaczącą różnicę widać jednak w 29 iteracji pętli, czyli później niż w poprzednim zadaniu.
Oznacza to, że liczby zmiennoprzecinkowe mogą mieć różną dokładność obliczeń w zależności od wykonywanych obliczeń matematycznych i sposobu ich zapisu w kodzie programu.
