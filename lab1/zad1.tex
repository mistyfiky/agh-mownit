\paragraph{Treść}~\\
Napisać program liczący kolejne wyrazy ciągu:\\
$ x_{n+1} = x_{n} + 3.0 * x_{n} * (1 - x_{n}) $\\
startując z punktu $ x_{0} = 0.01 $.
Wykonać to zadanie dla różnych reprezentacji liczb (\textit{float}, \textit{double}).
Dlaczego wyniki się rozbiegają?\\
\textit{Uwaga: Należy wprowadzić zmienne pomocnicze, aby uniknąć obliczeń w rejestrach procesora.}

\lstinputlisting[caption=Kod programu do zadania 1,label=lst:KodProgramuDoZadania1]{zad1.c}

\begin{table}[h]
  \caption{Wynik programu do zadania 1}
  \label{tab:WynikProgramuDoZadania1}
  \centering
  \begin{tabular}{r|c|c}
    n & float & double\\
    \hline 0 & 0.010000 & 0.010000\\
    \hline 1 & 0.039700 & 0.039700\\
    \hline 2 & 0.154072 & 0.154072\\
    \hline 3 & 0.545073 & 0.545073\\
    \hline 4 & 1.288978 & 1.288978\\
    \hline 5 & 0.171519 & 0.171519\\
    \hline \textbf{6} & \textbf{0.597819} & \textbf{0.597820}\\
    \hline 7 & 1.319113 & 1.319114\\
    \hline ... & ... & ...\\
    \hline 26 & 0.985675 & 0.070035\\
    \hline \textbf{27} & \textbf{1.028035} & \textbf{0.265426}\\
    \hline 28 & 0.941572 & 0.850352\\
    \hline ... & ... & ...\\
  \end{tabular}
\end{table}

\paragraph{Podsumowanie}~\\
Kod programu do zadania 1 zaprezentowany jest na listingu~\ref{lst:KodProgramuDoZadania1}.
Liczy on 99 kolejnych liczb ze wzoru podanego w treści zadania.
Fragmenty z wyników jego działania widać w tabeli~\ref{tab:WynikProgramuDoZadania1}.
Jako że funkcje programu operują na dwóch różnych typach danych \textit{float} oraz \textit{double} wyniki różnią się.
Pierwszą nieznaczną różnicę można zauważyć już w 6 iteracji pętli.
Później wyniki coraz bardziej się rozbiegają.
Znaczącą różnicę widać w 27 iteracji pętli, gdzie jedna wartość jest większa od 1, a druga od niej mniejsza.
Wynika to z faktu, że \textit{float} jest liczbą pojedynczej precyzji reprezentowaną według standardu IEEE 754~\cite{WikiIEEE754} przez 32 bity (w tym 1 bit na znak, 8 bitów na wykładnik oraz 23 bity na mantysę), a \textit{double} jest liczą podwójnej precyzji (składającej się według tego samego standardu z 1 bitu znaku, 11 bitów wykładnika oraz 52 bitów mantysy).
Przez inny zapis liczby podwójnej precyzji reprezentują około 16 dziesiętnych miejsc znaczących, natomiast pojedynczej tylko od 7 do 8.
Stąd też widoczne różnice w wyniku.

\begin{thebibliography}{9}
  \bibitem{WikiIEEE754} IEEE 754 - Wikipedia, wolna encyklopedia https://pl.wikipedia.org/wiki/IEEE\_754
\end{thebibliography}
