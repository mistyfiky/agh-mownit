\paragraph{Treść}~\\
Proszę skompilować i uruchomić program \textit{interpolacja.c}.
Korzystając z programu gnuplot narysować wykres.\\
Narysować na jednym wykresie krzywe otrzymane różnymi metodami interpolacji (w przykładzie ustawione jest \textit{gsl\_interp\_polynomial}).

\lstinputlisting[caption=Szkielet programu do zadań gsl,label=lst:InterpolacjaCSkel]{interpolacja.c.skel}

\lstinputlisting[caption=Kod programu do zadań gsl,label=lst:InterpolacjaC]{interpolacja.c}

\lstinputlisting[caption=Komendy dla programu gnuplot do zadań gsl,label=lst:InterpolacjaTxt]{interpolacja.txt}

\begin{figure}[p]
  \caption{Wynik programu gnuplot na podstawie danych z programu do zadań gsl}
  \label{fig:RysJpg}
  \centering
  \includegraphics[width=\textwidth]{rys.jpg}
\end{figure}

\paragraph{Podsumowanie}~\\
Kod programu do zadań gsl widoczny na listowaniu~\ref{lst:InterpolacjaC} został przygotowany na podstawie szablonu dołączonego do zadania widocznego na listowaniu~\ref{lst:InterpolacjaCSkel}.
Generuje on pliki dla wszystkich dostępnych typów interpolacji w bibliotece gnuplot~\cite{GslManualInterpolation}, które następnie przekazywane są do progamu gnuplot, poprzez załadowanie do niego komend z listowania~\ref{lst:InterpolacjaTxt}.
Wynik zaprezentowany jest na rysunku~\ref{fig:RysJpg}.

\begin{thebibliography}{9}
  \bibitem{GslManualInterpolation} GNU Scientific Library - Reference Manual: Interpolation https://www.gnu.org/software/gsl/manual/html\_node/Interpolation.html
\end{thebibliography}
