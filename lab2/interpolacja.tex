\paragraph{Treść}~\\
Proszę skompilować i uruchomić program \textit{interpolacja.c}.
Korzystając z programu gnuplot narysować wykres.\\
Narysować na jednym wykresie krzywe otrzymane różnymi metodami interpolacji (w przykładzie ustawione jest \textit{gsl\_interp\_polynomial}).

\lstinputlisting[caption=Kod programu do zadań gsl,label=lst:KodProgramuDoZadanGsl]{interpolacja.c}

\begin{figure}[h]
  \caption{Wynik programu gnuplot na podstawie danych z programu do zadań gsl}
  \label{fig:WynikProgramuGnuplotNaPodstawieDanychZProgramuDoZadanGsl}
  \centering
  \includegraphics[width=\textwidth]{rys.jpg}
\end{figure}

\paragraph{Podsumowanie}~\\
Wynik dla zadania zaprezentowany jest na rysunku~\ref{fig:WynikProgramuGnuplotNaPodstawieDanychZProgramuDoZadanGsl}.
Został on przygotowany na podstawie szablonu dołączonego do zadania.
Rysunek został przygotowany przez program gnuplot ze wszystkich dostępnych typów interpolacji~\cite{GslManualInterpolation}.

\begin{thebibliography}{9}
  \bibitem{GslManualInterpolation} GNU Scientific Library - Reference Manual: Interpolation https://www.gnu.org/software/gsl/manual/html\_node/Interpolation.html
\end{thebibliography}
