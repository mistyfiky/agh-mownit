\paragraph{Treść} ~\\
Proszę skompilować i uruchomić program \textit{interpolacja.c}.
Korzystając z programu gnuplot narysować wykres. \\
Narysować na jednym wykresie krzywe otrzymane różnymi metodami interpolacji (w przykładzie ustawione jest \textit{gsl\_interp\_polynomial})\nocite{GslManualInterpolation}.

\paragraph{Rozwiązanie} ~\\
\lstinputlisting[caption={Kod programu do zadań gsl},label=lst:KodProgramuDoZadanGsl]{interpolacja.c} ~\\

\paragraph{Wynik} ~\\
\begin{figure}[h]
  \centering
  \includegraphics[width=\textwidth]{rys.jpg}
  \caption{Wynik programu gnuplot na podstawie danych z programu do zadań gsl}
  \label{fig:WynikProgramuGnuplotNaPodstawieDanychZProgramuDoZadanGsl}
\end{figure}

\paragraph{Podsumowanie} ~\\

\begin{thebibliography}{9}
  \bibitem{GslManualInterpolation} GNU Scientific Library - Reference Manual: Interpolation https://www.gnu.org/software/gsl/manual/html\_node/Interpolation.html
\end{thebibliography}
