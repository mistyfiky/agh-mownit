\section*{Zadania gsl} \label{sec:Zadania gsl}

\paragraph{Treść} ~\\
Proszę skompilować i uruchomić program \textit{interpolacja.c}.
Korzystając z programu gnuplot narysować wykres. \\
Narysować na jednym wykresie krzywe otrzymane różnymi metodami interpolacji (w przykładzie ustawione jest \textit{gsl\_interp\_polynomial})\nocite{GslManualInterpolation}.

\paragraph{Rozwiązanie} ~\\
\lstinputlisting[language=C,caption={Kod programu do zadań gsl},captionpos=b,label=lst:KodProgramuDoZadanGsl]{interpolacja.c} ~\\

\paragraph{Wynik} ~\\

\paragraph{Podsumowanie} ~\\

\begin{thebibliography}{9}
  \bibitem{GslManualInterpolation} GNU Scientific Library - Reference Manual: Interpolation https://www.gnu.org/software/gsl/manual/html\_node/Interpolation.html
\end{thebibliography}
