\paragraph{Treść}~\\
Przy pomocy gnuplot proszę narysować dane zgromadzone w pliku \textit{dane1.dat}.
Aby wykres był czytelny, jedna z osi musi mieć skalę logarytmiczną.
Prosze ustalić, która to oś i narysowac wykres.

\lstinputlisting[caption=Zawartosc pliku dane1.dat,label=lst:ZawartoscPlikuDane1Dat]{dane1.dat}

\lstinputlisting[caption=Komendy dla programu gnuplot do zadania 1 gnuplot,label=lst:KomendyDlaProgramuGnuplotDoZadani1Gnuplot]{dane1.txt}

\begin{figure}[h]
  \caption{Wynik programu gnuplot na podstawie zawartości pliku dane1.dat}
  \label{fig:WynikProgramuGnuplotNaPodstawieZawartosciPlikuDane1Dat}
  \centering
  \includegraphics[width=\textwidth]{dane1.jpg}
\end{figure}

\paragraph{Podsumowanie}~\\
Wynik generowany jest poprzez załadowanie do programu gnuplot komend z listowania~\ref{lst:KomendyDlaProgramuGnuplotDoZadanGsl}.
Operują one na danych z pliku załączonego do zadnia, którego zawartość widoczna jest na listingu~\ref{lst:ZawartoscPlikuDane1Dat}.
Aby wykres był czytelny oś x ma skalę logarytmiczną.
Wynik zaprezentowany został na rysunku~\ref{fig:WynikProgramuGnuplotNaPodstawieZawartosciPlikuDane1Dat}.
