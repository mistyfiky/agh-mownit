\paragraph{Treść}~\\
Proszę narysować wykres funkcji dwywymiarowej, której punkty znajdują się w pliku \textit{dane2.dat}.
Proszę przegladnać plik i spróbować znaleźć w nim maksimum.
Potem proszę zlokalizować maksimum wizualnie na wykresie.
Proszę na wykresie zaznaczyć maksimum strzałką.

\lstinputlisting[caption=Początek pliku dane2.dat,label=lst:Dane2Dat,lastline=10]{dane2.dat}

\lstinputlisting[caption=Komendy dla programu gnuplot do zadania 2,label=lst:Dane2Txt]{dane2.txt}

\begin{figure}[h]
  \caption{Wynik programu gnuplot na podstawie zawartości pliku dane2.dat}
  \label{fig:Dane2Jpg}
  \centering
  \includegraphics[width=\textwidth]{dane2.jpg}
\end{figure}

\paragraph{Podsumowanie}~\\
Wynik generowany jest poprzez załadowanie do programu gnuplot komend z listowania~\ref{lst:Dane2Dat}.
Operują one na danych z pliku załączonego do zadnia, którego początek zawartości widoczna jest na listingu~\ref{lst:Dane2Txt}.
Wynik zaprezentowany został na rysunku~\ref{fig:Dane2Jpg}.
Zgodnie z poleceniem została dodana strzałka w punkcie (4,3), gdzie znajduje się maksimum funkcji w dziedzinie ograniczonej danymi o wartości 1.
