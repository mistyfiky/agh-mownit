\paragraph{Treść}~\\
Proszę narysować wykres funkcji dwywymiarowej, której punkty znajdują się w pliku \textit{dane2.dat}.
Proszę przegladnać plik i spróbować znaleźć w nim maksimum.
Potem proszę zlokalizować maksimum wizualnie na wykresie.
Proszę na wykresie zaznaczyć maksimum strzałką.

\lstinputlisting[caption=Zawartosc pliku dane2.dat,label=lst:ZawartoscPlikuDane2Dat]{dane2.dat}

\lstinputlisting[caption=Komendy dla programu gnuplot do zadania 2,label=lst:KomendyDlaProgramuGnuplotDoZadania2]{dane2.txt}

\begin{figure}[h]
  \caption{Wynik programu gnuplot na podstawie zawartości pliku dane2.dat}
  \label{fig:WynikProgramuGnuplotNaPodstawieZawartosciPlikuDane2Dat}
  \centering
  \includegraphics[width=\textwidth]{dane2.jpg}
\end{figure}

\paragraph{Podsumowanie}~\\
Wynik generowany jest poprzez załadowanie do programu gnuplot komend z listowania~\ref{lst:KomendyDlaProgramuGnuplotDoZadania2}.
Operują one na danych z pliku załączonego do zadnia, którego zawartość widoczna jest na listingu~\ref{lst:ZawartoscPlikuDane2Dat}.
Wynik zaprezentowany został na rysunku~\ref{fig:WynikProgramuGnuplotNaPodstawieZawartosciPlikuDane2Dat}.
