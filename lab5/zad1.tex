\paragraph{Treść}~\\
Uruchomić program root\_finding.tgz.\\
- Umieć odpowiedzieć na pytanie, co on robi.\\
- Narysować (np. za pomocą gnuplota) wykres funkcji, której miejsc zerowych szukamy.

\lstinputlisting[caption=Funkcje dołączone do zadania 1,label=lst:DemoFnC]{demo_fn.c}

\lstinputlisting[caption=Program dołączony do zadania 1,label=lst:RootFindingC]{root_finding.c}

\lstinputlisting[caption=Komendy dla programu gnuplot do zadania 1,label=lst:Zad1Txt]{zad1.txt}

\begin{figure}[p]
  \caption{Wynik programu gnuplot na podstawie komend do zadania 1}
  \label{fig:Zad1Jpg}
  \centering
  \includegraphics[width=\textwidth]{zad1.jpg}
\end{figure}

\paragraph{Podsumowanie}~\\
Wykres~\ref{fig:Zad1Jpg}, został wygenerowany w programie gnuplot poprzez załadowanie komend z listingu~\ref{lst:Zad1Txt}.
Prezentuje on przebieg funkcji $ x^2 - 5 $, której miejsca zerowe są obliczane w programie z listingu~\ref{lst:DemoFnC} oraz ~\ref{lst:RootFindingC}.
Program szuka pierwiastka funkcji kwagratowej w podanym przedziale.
Działa on na zasadzie połowicznego dzielenia przedziału oraz odrzuceniu tego podprzedziału, w którym nie występuje zmiana znaku między końcami.
Środek wybranego przedziału jest aktualnym przybliżeniem pierwiastka.
